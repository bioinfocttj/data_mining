\section*{Résultats}
\addcontentsline{toc}{chapter}{Résultats}

\subsection*{Résultats de la clusterisation}
\addcontentsline{toc}{section}{Résultats de la clusterisation}
Après clusterisation de la taille, on obtient 17 clusters.\\
Après clusterisation de la structure secondaire, on obtient 68 clusters.\\
Après clusterisation de l'hydrophobicité, on obtient 197 clusters.\\
Après clusterisation du pHi, on obtient 467 clusters.\\
Après clusterisation des cystéines, on obtient 1022 clusters.\\
Après clusterisation de la localisation, on obtient 1929 clusters.\\

Exemple de cluster :
\begin{verbatim}
cluster n°1506
	-Pyruvate kinase isozymes M1/M2
	-NADP-dependent malic enzyme
	-Dihydropyrimidinase-related protein 2
	-Beta-hexosaminidase subunit beta
\end{verbatim}
\subsection*{Analyse et critique des résultats}
\addcontentsline{toc}{section}{Analyse des résultats}
Après avoir déterminé les moyennes et écart-types de chaque cluster, nous nous sommes rendus compte que les écart-types étaient relativement corrects. En effet, si l'on prend l'exemple de l'hydrophobicité dont les valeurs sont comprises entre -4,5 et 4,5 (calcul de l'hydrophobicité par la méthode GRAVY), on retrouve un écart-type maximal proche de 0,4. Cependant, la justesse de ces résultats est toute relative car on retrouve de nombreux clusters contenant une seule protéine.\\
Ce problème se retrouve à tous les niveaux de la clusterisation et plus particulièrement au dernier où l'on retrouve 1221 clusters mono-protéiques.\\

Nous avons envisagé la possibilité de ne pas réaliser ce dernier niveau de clusterisation afin de réduire ces clusters mono-protéiques (500 à l'avant dernier niveau), mais certains clusters deviennnent alors énormes (plus de 100 protéines). Il aurait fallu, au niveau des tissus, dans le cas des protéines sans données:
\begin{itemize}
\item Ne pas les prendre en compte lors de la fabrication du data warehouse, mais cela réduisait trop le nombre de données,
\item ou arr\^eter la clusterisation des ces protéines au niveau précédent.\\
\end{itemize}

Par ailleurs, la stratégie de clusterisation que nous avons appliqué n'élimine pas les protéines hors-champ (out layers), elle fausse donc en partie les résultats obtenus.



