\section*{Création de notre data warehouse}
\addcontentsline{toc}{chapter}{Création de notre data warehouse}

\subsection*{Choix et récupération des données}

\addcontentsline{toc}{section}{Choix et récupération des données}

Afin de répondre à la problématique de départ, nous avons décidé de nous intéresser uniquement aux protéines humaines ayant été publiée, et plus particulièrement celle disponible sur la base de données \emph{uniprot}.

En effet, la requête [\emph{(homo sapiens) AND reviewed:yes}] sur cette base de données permet d'obtenir 26,070 protéines, ce qui est suffisant pour obtenir des résultats exploitable sans que cela ne devienne trop chronophage en terme de temps de traitement et d'analyse.

Nous récupérons donc les données sous formes de fichier XML nous permettant de lire et de traiter facilement les données (parse de fichier texte). Les balises propre au système xml (<balise> information <\textbackslash balise>) facilitant la récupération des données d'inter\^et.

\subsection*{Préparation des données}

\addcontentsline{toc}{section}{Préparation des données}
La préparation des données est indispensable car les données récupérées peuvent être incomplètes, bruitées ou incohérentes. 
Dans un premier temps, nous avons sélectionnés les critères que nous prendrons en compte dans notre analyse :
\renewcommand\labelitemi{\textbullet}
\begin{itemize}
\item accession number (id)
\item name/full name
\item tissue
\item sequence length
\item gene - primary
\item feature type ="strand  - helix - turn"
\end{itemize}

Cette récupération se fait grâce à un programme python que nous avons écrit nous permettant de "parser" les fichiers XML précédemment acquis.

Nous obtenons ainsi des fichiers XML contenant uniquement les données que nous avons estimé nécéssaire au traitement du sujet.

\pagebreak
\subsubsection*{Cluster}
\addcontentsline{toc}{subsection}{Cluster}

\begin{enumerate}
\item conformation (nombre d'hélices et de feuillets)
\item taille de la chaine (en nombre de acides aminés)
\item localisation au niveau organe
\item pourcentage d'hydrophobicité
\item nombre de cystéines
\item pHi
\item gène codant la protéine
\end{enumerate}

\subsubsection*{Nettoyage}
\addcontentsline{toc}{subsection}{Nettoyage}
