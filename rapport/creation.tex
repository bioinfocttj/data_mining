\section*{Création de notre data warehouse}
\addcontentsline{toc}{chapter}{Création de notre data warehouse}

\subsection*{Choix et récupération des données}

\addcontentsline{toc}{section}{Choix et récupération des données}

Afin de répondre à la problématique de départ, nous avons décidé de nous intéresser uniquement aux protéines humaines ayant été publiées, et plus particulièrement à celles disponibles sur la base de données \emph{uniprot}.

En effet, la requête [\emph{(homo sapiens) AND reviewed:yes}] sur cette base de données permet d'obtenir 26070 protéines, ce qui est suffisant pour obtenir des résultats exploitables sans que cela ne devienne trop chronophage en terme de temps de traitement et d'analyse.

Nous récupérons donc les données sous la forme d'un fichier XML nous permettant de lire et de traiter facilement les données (parse de fichier texte). Les balises propres au système xml (<balise> information <\textbackslash balise>) facilitant la récupération des données d'inter\^et.

\subsection*{Préparation des données}

\addcontentsline{toc}{section}{Préparation des données}
La préparation des données est indispensable car les données récupérées peuvent être incomplètes, bruitées ou incohérentes. 

\subsubsection*{Nettoyage}
\addcontentsline{toc}{subsection}{Nettoyage}
Dans un premier temps, nous avons sélectionné les balises correspondantes aux critères que nous prendrons en compte dans notre analyse :
\renewcommand\labelitemi{\textbullet}
\begin{itemize}
\item accession number (id)
\item name/full name
\item tissue
\item sequence length
\item feature type ="strand  - helix - turn"\\
\end{itemize}

Cette récupération se fait grâce à un programme python que nous avons écrit nous permettant de "parser" le fichier XML précédemment acquis.\\

Ce script comporte deux parties, nous supprimons d'abord les balises qui ne sont pas utiles pour notre étude.
Nous ajoutons ensuite des données calculées au fichier précédent qui vont nous permettre de créer les clusters par la suite. Il s'agit de:
\begin{itemize}
\item Le nombre et le pourcentage d'hélices, de feuillets et de coudes dans la protéine.
\item Le pourcentage d'hydrophobicité.
\item Le pHi (point isoélectrique) qui a été déterminé à l'aide de la fonction \texttt{isoelectric\_point()} présente dans BioPython.
\item Le nombre de cystéines.
\end{itemize}

Nous avons également supprimé les informations dupliquées (identifiants, noms etc).
Enfin nous avons remarqué que certaines protéines n'étaient pas des protéines humaines mais appartenaient à des virus, nous les avons donc supprimé.

Nous obtenons ainsi un fichier XML contenant uniquement les données que nous avons e


\subsubsection*{Cluster}
\addcontentsline{toc}{subsection}{Cluster}
Nous avons choisi de réaliser un clustering hierarchique car l'attribution de scores aux protéines de le cas du clustering par scores nous paraissait fastidieuse.\\
Ainsi, nous avons choisi 7 niveaux de clusterisation:
\begin{enumerate}
\item Taille de la chaine (en nombre de acides aminés)
\item Conformation (nombre d'hélices, de feuillets, de coudes et de cystéines)
\item Localisation au niveau organe
\item Pourcentage d'hydrophobicité
\item pHi
\end{enumerate}


\subsubsection*{Taille de la chaîne}
Les protéines sont constitués d'une succession d'acides aminés reliés entre eux par des liaisons peptidiques. La taille d'une protéine correspond au nombre d'acides aminés qu'elle contient. La taille des protéines est très variable, des protéines ayant des tailles très différentes n'auront pas les m\^emes proriétés.

Pour déterminer la distance entre deux protéines, nous calculons la distance de Manhattan.



\subsubsection*{Localisation}

\subsubsection*{Conformation}
Une protéine peut être décrite par un enchaînement de structures secondaires (hélices alpha, feuillets beta et coude) qui ont une influence sur son repliement. 

\subsubsection*{Nombre de cystéine}
Les ponts disulfures se forment au sein d'une protéine entre deux cystéines et créent des liaisons inter-chaines. 


\subsubsection*{Hydrophobicité}
L'hydrophobicité d'une protéine est déterminée à partir du nombre d'acides aminés hydrophobes qu'elle possède.
Nous avons calculé un index d'hydrophobicité (GRAVY: Grand average of hydropathicity index,Kyte and Doolittle) qui indique la solubilité d'une protéine. Si la valeur est positive, la protéine est hydrophobe, sinon elle est hydrophile.
Son caractère hydrophobe ou hydrophile influence :
\begin{itemize}
\item sa localisation au niveau cellulaire (cytoplasme ou membrane)
\item son repliement (acides aminés hydrophobes à l'intérieur ou l'extérieur de la protéine selon sa localisation cellulaire) 
\end{itemize}

\subsubsection*{pHi}
Le pHi est le pH isoéléctrique d'une protéine, c'est à dire le pH auquel cette protéine a une charge nulle.
Au dessus de ce pH, la protéine est chargée négativement et inversement.

